\documentclass{article}

\title{MetaGen Inference}
\author{Zhangir Azerbayev}
\date{2022}

\usepackage{babel}   % multi-language support
\usepackage{float}   % floats
\usepackage{url}     % urls

\usepackage[l2tabu,orthodox]{nag}  % force newer (and safer) LaTeX commands
\usepackage[utf8]{inputenc}        % set character set to support some UTF-8
                                   %   (unicode). Do NOT use this with
                                   %   XeTeX/LuaTeX!
\usepackage{sectsty}               % allow redefinition of section command formatting
\usepackage{tabularx}              % more table options
\usepackage{titling}               % allow redefinition of title formatting
\usepackage{imakeidx}              % create and index of words
\usepackage{xcolor}                % more colour options
\usepackage{enumitem}              % more list formatting options
\usepackage{tocloft}               % redefine table of contents, new list like objects

\usepackage{parskip}
\usepackage{fullpage}


\usepackage{amsmath}                                    % extensive math options
\usepackage{amssymb}                                    % special math symbols
% \usepackage[Gray,squaren,thinqspace,thinspace]{SIunits} % elegant units
\usepackage{listings}                                   % source code

\providecommand{\abs}[1]{\left\lvert#1\right\rvert}                    % |.|
\providecommand{\pr}[1]{\left(#1\right)}                               % (.)
\providecommand{\br}[1]{\left[#1\right]}                              % [.]
\providecommand{\cbr}[1]{\left\{#1\right\}}


% options for listings
\lstset{
  breaklines=true,
  postbreak=\raisebox{0ex}[0ex][0ex]{\ensuremath{\color{red}\hookrightarrow\space}},
  numbers=left,
  numbersep=5pt,
  numberstyle=\tiny\color{gray},
  basicstyle=\footnotesize\ttfamily
}



% format hyperlinks
\colorlet{linkcolour}{black}
\colorlet{urlcolour}{blue}
% \hypersetup{colorlinks=true,
%            linkcolor=linkcolour,
%            citecolor=linkcolour,
%            urlcolor=urlcolour}




\begin{document}
\maketitle

\section{Inference Framework}
In our baseline model, we have $N$ observations $X = \cbr{x^n}$, where each $x^n$ is a coordinate in 3-space. In the
full model, we have $N$ observations $X = \cbr{(x^n, c^n)}$, where the $x^n$ are locations and the $c_n$ are object
categories. We are trying to infer the world-state $\Theta = \cbr{(\mu_k, c_k)}_{k=1}^K$, where the $\mu_k$ are object
locations and the $c_k$ are object categories. 

Suppose we augment are model with a latent variable $Z = \{z_k^n\}$, where $z^n$ is a one-hot vector that describes
which compnent of the GMM generated $X^n$. Thus $P(X|\Theta)$ factors as $P(X|\Theta) = P(Z)P(X|Z, \Theta)$. 

The {\it expectation-maximization} (EM) algorithm hinges on two criteria: 
\begin{enumerate}
    \item It is easy to optimize $P(X, Z|\Theta)$ wrt $\Theta$ (see later section)
    \item Although we have poor ``global" information about $Z$, i.e we don't know $P(Z|X)$, we have ``local"
        information about $Z$, i.e we know $P(Z|X, \Theta)$ for a fixed $\Theta$. 
\end{enumerate}
The idea of the algorithm is this: instead of maximizing log-likelihood, maximize the expectation of log-likelihood
under the posterior of $Z$. More precisely, we iterate the two following steps: 
\begin{enumerate}
    \item E-step: Using $\Theta^{\mathrm{old}}$ from the previous iteration, evaluate $P(Z|X, \Theta^{\mathrm{old}})$. 
    \item M-Step: Solve for $\Theta = \mathrm{argmax}_{\Theta} \sum_Z P(Z|X, \Theta)\log P(X, Z|\Theta^{\mathrm{old}})$, which
        is the expectation of log-likelihood under the posterior of $Z$. 
\end{enumerate}
\section{Inference for MetaGen}
Define the {\it responsibility} of $z_k^n$ for $x^n$ as 
\[\gamma(z_k^n) = P(z_k^n=1|x^n, \Theta) = \frac{P(x^n|z_k^n=1, \Theta)}{\sum_{j=1}^KP(x^n|z_j^n, \Theta)}. \]
Note that the E-step is equivalent to computing the responsibilities for each $n$ and $k$. The M-step then becomes 
\begin{equation}\mathrm{argmax}_\Theta\sum_{n=1}^N\sum_{k=1}^K\gamma(z_k^n)\log P(x^n, z_k^n=1| \Theta^{\mathrm{old}}).
\end{equation}
In the case of the baseline model, this is a continuous optimization problem. In the full MetaGen case, we can simply
sample category assignments and condition on them, yielding multiple continuous optimization problems. 

\section{Optimization for M-step}
I currently do not know whether (1) has a closed form solution (my guess is that it doesn't). Worst case, we do
gradient-based optimizaton. 
\end{document}
